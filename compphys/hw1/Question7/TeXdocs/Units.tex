\documentclass[12pt, oneside]{article}   	
\usepackage[margin=1in]{geometry}                		
%set the margins
\geometry{letterpaper}                   		
%choose the paper
\usepackage[parfill]{parskip}    		
%new paragraphs separated by newline
\usepackage{amssymb}
%some mathematical symbols
\usepackage{graphicx}
%be able to draw figures
\usepackage{bm}
%boldface for greek symbols
\usepackage{wrapfig}
%this allows text to flow around figures
\usepackage{amsmath}
\usepackage{float}

\begin{document}
	
\section*{Units Derivation}	
	In setting \(\hbar=1\) and \(m=m_e\) our units for distance and time must effectively carry the units required to make this substitution happen. Our distance parameter x must then contain the inverse of \(\frac{\hbar^2}{2m}\rightarrow\frac{eV^2 s^2}{m_e} \) and likewise our time units must contain the inverse of \(\hbar\rightarrow \frac{1}{eV s}\). Together with their natural units of distance and time this requires our units
	\begin{equation*}
	[x]=\frac{m_e\cdot m}{eV^2s^2}, [t]=\frac{1}{eV}.
	\end{equation*}
	
	Expressed in standard units this relationship is,
		\begin{equation*}
	[x]=\frac{s^2}{m^3kg}, [t]=\frac{s^3}{m^2kg}.
	\end{equation*}
	
\end{document}