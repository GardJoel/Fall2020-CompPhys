\documentclass[12pt, oneside]{article}   	
\usepackage[margin=1in]{geometry}                		
%set the margins
\geometry{letterpaper}                   		
%choose the paper
\usepackage[parfill]{parskip}    		
%new paragraphs separated by newline
\usepackage{amssymb}
%some mathematical symbols
\usepackage{graphicx}
%be able to draw figures
\usepackage{bm}
%boldface for greek symbols
\usepackage{wrapfig}
%this allows text to flow around figures
\usepackage{amsmath}
\usepackage{float}

\begin{document}
	\section*{Analytic Solution}
	
	The given logistic map is
	\begin{equation*}
	x_{n+1}=r\frac{x_n}{1+x_n^2}.
	\end{equation*}
	Assuming there is a stable fixed point such that \(x_{n+1}=x_{n}=x_s\) and inserting this into the logistic map yields
	\begin{equation*}
	x_s=\pm\sqrt{r-1}.
	\end{equation*}
	In class it was shown for an arbitrary logistic map \(f(x)\) the condition \(|f'(x_s)|<1\) defines a region of stability. Appling this condition gives \(|\frac{r}{1+x_s^2}-\frac{2rx_s^2}{(1+x_s^2)^2}|<1\). Substituting \(x_s\) into this relation and disributing the absolute value leaves the inequality
	\begin{equation*}
	-1<1-\frac{2(r^2-r)}{r^2}<1\rightarrow 0>-\frac{1}{r}>-1\rightarrow r>1.
	\end{equation*}
\section*{Comments on B and C}

\begin{figure}[H]
	\centering
	\includegraphics[width=60mm,height=60mm]{"bifurcation_b".png}
	\label{0}
	\caption{Bifurcation Diagram \(x_{n+1}=r\frac{x_n}{1+x_n^2}\)}
	\end{figure}
	As can be seen in Figure 1, the logistic map is stable in the range \(1\geq r \leq 15\). This agrees with the analytic solution attained above. There is no apparent period doubling.
\begin{figure}[H]
	\centering
	\includegraphics[width=60mm,height=60mm]{"bifurcation_c".png}
	\label{0}
	\caption{Bifurcation Diagram \(x_{n+1}=r\frac{x_n}{1+x_n^4}\)}
\end{figure}	
As seen in figure 2 there period doubling which occurs at \(r=2\) and further period multiplicity which reduces at \(r=7 \) before ramping up again.

\end{document}